\documentclass{../semdoc}
% Template: $Id: t01_txt.tex,v 1.7 2000/05/23 12:13:37 bless Exp $
% -----------------------------------------------------------------------------
%epstopdf ermöglicht, dass eps-Dateien durch pdflatex in windows eingebunden werden können
\usepackage{epstopdf}
\usepackage{xcolor}
\usepackage{listings}
\usepackage{cleveref}


% Credits to https://latexcolor.com/ <3
\definecolor{amaranth}{rgb}{0.9, 0.17, 0.31}
\definecolor{airforceblue}{rgb}{0.36, 0.54, 0.66}\definecolor{anti-flashwhite}{rgb}{0.95, 0.95, 0.96}

\newcommand*{\Comment}[1]{\hfill\makebox[3.0cm][l]{#1}}%
\lstset{language=bash,
    backgroundcolor=\color{anti-flashwhite},
    commentstyle=\color{amaranth},
    %keywordstyle=\color{airforceblue},
    basicstyle=\ttfamily\small,
	captionpos=b,
	breaklines=true, 
  	escapechar=\&% char to escape out of listings and back to LaTeX
}

% Report Praktikum 
% -----------------------------------------------------------------------------
% Kommentare beginnen mit einem %-Zeichen
\docbegin
% --> Oberhalb der Linie bitte nichts ändern.
% ---------------------------------------------------------------------------
% \/ \/ \/ \/ \/ \/ \/ \/ \/ \/ \/ \/ \/ \/ \/ \/ \/ \/ \/ \/ \/ \/ \/ \/ \/ 
% Stellen, an denen etwas geaendert werden soll, sind wie hier gekennzeichnet.
% /\ /\ /\ /\ /\ /\ /\ /\ /\ /\ /\ /\ /\ /\ /\ /\ /\ /\ /\ /\ /\ /\ /\ /\ /\ 

%
% ---------------------------------------------------------------------------
% \/ \/ \/ \/ \/ \/ \/ \/ \/ \/ \/ \/ \/ \/ \/ \/ \/ \/ \/ \/ \/ \/ \/ \/ \/ 
% --> Bitte den Titel des Beitrages in die nächste Zeile eintragen:
\title{Intro - Reference Monitor (SELinux)}
%
% --> ... und den Namen des Autors:
\author{Carolin Beer - 1842604}
% /\ /\ /\ /\ /\ /\ /\ /\ /\ /\ /\ /\ /\ /\ /\ /\ /\ /\ /\ /\ /\ /\ /\ /\ /\ 
% -----------------------------------------------------------------------------

% Nicht ändern!
\event{Access Control Systems Lab\\}
\term{Sommersemester 2020}
\supervisor{Prof. Dr. Hannes Hartenstein, Jan Grashöfer, Florian Jacob}

%
%
\maketitle

\section{Introduction}
Security-Enhanced Linux (SELinux) is a \emph{Linux Security Module (LSM)} which implements Mandatory Access Control (MAC) for Linux kernels. It has become a crucial part of Access Control systems and is part of most recent Linux distributions. In the following, we outline the history of SELinux in \cref{sec:background} and explain the basic concepts behind it in \cref{basics}. Afterwards, the theoretical results are linked to practice in \cref{sec:using}. 

\section{History and Background of SELinux} % in die Klammern die Ueberschrift 
\label{sec:background}
%SELinux is a well-established LSM providing complex access control functionality to Linux. In
%particular, SELinux can be used to implement Mandatory Access Control (MAC). Have a look
%at its history and the project’s background to answer the following questions:
%	a) Under which license is SELinux available?
%	b) Who are the most important contributers?
%	c) Who is in charge of SELinux?

 %https://github.com/containers/container-selinux/blob/master/LICENSE
It has initially been developed by the National Security Agency (NSA) and was made open source under the \emph{GNU General Public License (GPL)} in 2000. 
Among others, major contributors to SELinux include Red Hat and Network Associates (today known as McAfee).
At the time of writing, upstream development of SELinux is largely driven by RedHat and coordinated through GitHub. Almost 100 collaborators have already contributed to the code base. 


\section{Basic Concepts of SELinux}
\label{basics}

%SELinux is quite complex, thus we will not dive too deep into its internals. Nevertheless, many
%popular Linux distributions ship with SELinux. Therefore a basic understanding of its concepts
%is important. Make yourself familiar with the following concepts of SELinux and describe them:
%	a) SELinux policies
%	b) SELinux labels
%	c) SELinux modes
Most operating systems rely on \emph{Discretionary Access Control (DAC)} to implement access policies for users and objects. Every object is owned by a user, and users are in power of the access policies for their objects. Therefore, instead of consistent, global access control, DAC leads to fragmented policies that may even infringe system security. 
Importantly, a process that is run by some processes obtains the same access privileges as the callee, thus making the system vulnerable to malicious applications.

In contrast, SELinux implements MAC and defines the rights of every user, process, and object on the system separately. Importantly, SELinux does not substitute DAC, it works on top of it.

In the following, the core concepts of SELinux, namely \emph{policies}, \emph{labels} and \emph{modes} are explained.
%Every user or process is assigned a context of username, role and domain or type by SELinux. 

\subsection{SELinux policies}
Policies in SELinux are used to decide whether operations are authorized or not. By design philosophy, policies should only grant access where needed to preserve functionality, but ultimately, it is up the system administrator to define them. 


Each process or object has a \emph{security context} consisting of \emph{name}, \emph{role} and \emph{type}, which is directly affecting its permissions in the system.
The type is an attribute defining the \emph{domain} for processes or type for objects. The policy defines how types can access each other. 

Users are authorized for a set of roles by the policy. Roles, in turn authorize for specific domains. 
Users may launch processes in a specific security context, however the transition must be in accordance with the policy.


When an subject, such as a process, attempts to access an object, a policy enforcement server queries the permissions in an \emph{Access Vector Cache (AVC)}. Upon cache failure, the AVC queries a security server to look up the security context of the subject and object to decide whether access is granted. Denied access requests are written to a log file.



%"A typical policy consists of a mapping (labeling) file, a rule file, and an interface file, that define the domain transition. These three files must be compiled together with the SELinux tools to produce a single policy file. "
% Policies define \emph{types} for objects and \emph{domains} for processes. Depending on the \emph{role} of a user 
 
\subsection{SELinux labels}
The mapping between security context and processes and objects is called \emph{labelling} and is part of and defined by the policy. However, it is possible to make changes to this mapping without rewriting the entire policy.


\subsection{SELinux modes}
SELinux can run in two different modes, in the \emph{enforcing} or the \emph{permissive} mode or be \emph{disabled} entirely.
In the enforcing mode, SELinux enforces the implemented policy and restricts access. On the other hand, the permissive mode is useful for debugging and troubleshooting purposes. While the policy is evaluated, it merely displays warnings but does not deny resource access. 
Both modes are logging actions and write them to a file.


\section{Using SELinux}
\label{sec:using}
\subsection{\texttt{sestatus}}
\label{sec:sestatus}

\texttt{sestatus} is a command displaying information about the current status of SELinux. 

\begin{lstlisting}[caption={Output of \texttt{sestatus}},tabsize=4, label={sestatus}]
$ sestatus
SELinux status:                    enabled
SELinuxfs mount:                   /sys/fs/selinux
SELinux root directory:            /etc/selinux
Loaded policy name:                targeted
Current mode:                      enforcing
Mode from config file:             enforcing
Policy MLS status:                 enabled
Policy deny_unknown status:        allowed
Max kernel policy version:         31
\end{lstlisting}

In \cref{sestatus}, we can see an exemplary output when running the command. 
For the given system, we can see that SELinux is enabled. 
The filesystem mount point of SELinux is \texttt{/sys/fs/selinux}, this is the place where internal data, such as the AVC resides. 
In the root directory, \texttt{/etc/selinux}, configuration files are stored.
The current policy is set to \emph{targeted}, which is the default. This means, that only certain, targeted, processes are protected by SELinux.
The policy mode is set to enforcing. 
\emph{MLS} refers to Multi-Level Security, it assigns security levels to users, types and processes.
Furthermore, we can see that actions which are not specified by the policy, are permitted. 
The kernel policy version is 31. 

\subsection{Accessing files through a web server}
We are now starting a web server using \texttt{sudo systemctl start httpd}. Sending a \texttt{curl} request for the \texttt{index.html} file yields the content of the page stored at \texttt{/var/www/html/index.html}, as shown in \cref{file}.

\begin{lstlisting}[caption={Content of the index.html file}, label={file}]
$ sudo systemctl start httpd
$ curl http://localhost/index.html
Security = Access Control
\end{lstlisting}

%	b) Display the file attributes using ls -lZ and explain what you see.
%	c) Change the type label of the file to samba_share_t using chcon . Afterwards, try to retrieve
%	the file from the webserver again and explain the result.
%	Your previous access request should have failed. Access decisions made by SELinux are logged
%	by auditd to /var/log/audit/audit.log . Additionally, you will find failed access request in the
%	system’s journal (to view the journal use journalctl ).

Accessing the file attributes through \texttt{ls -lZ /var/www/html/index.html} gives insight into the security context. 

\begin{lstlisting}[caption={File attributes of the \texttt{index.html} file}, label={seccontext}]
$ ls -lZ /var/www/html/index.html
-rw-r--r--. 1 root root unconfined_u:object_r:httpd_sys_content_t:s0 
26 May 3 2017 /var/www/html/index.html
\end{lstlisting}

In \cref{seccontext} we can see the output of the command. The first part shows the access permissions imposed by DAC. The following quadruple, separated by colons, is the security context for the file. 
The syntax is \texttt{user:role:type:mls\_level}. 
We can see that the object belongs to the user \texttt{unconfined\_u}, which is the default in Linux. In the targeted policy, roles are only used for processes. 
Thus, \texttt{object\_r} is a placeholder, as the role has no meaning for objects. The \texttt{httpd\_sys\_content\_t} type grants the \texttt{httpd} process access to the object.
Finally, $s0$ is the security level that is assigned to the object through MLS.

\subsection{Changing of object types}
Now we change the type of the index file to \texttt{samba\_share\_t} using \texttt{chcon} and retry accessing it through the web server. 


\begin{lstlisting}[caption={Changing the type of the \texttt{index.html} file}, language=HTML, label={chcon}]
$ sudo chcon -t samba_share_t /var/www/html/index.html
$ curl http://localhost/index.html
<!DOCTYPE HTML PUBLIC "-//IETF//DTD HTML 2.0//EN">
<html><head>
<title>403 Forbidden</title>
</head><body>
<h1>Forbidden</h1>
<p>You don't have permission to access /index.html 
on this server.<br /> 
</p>
</body></html>
\end{lstlisting}

Clearly, access has been denied now. 

\subsection{Obtaining logged information}

As mentioned in \cref{basics}, denied access requests are logged by AVC. 
Looking at the log file \texttt{audit.log} in the folder \texttt{/var/log/audit/}. The output is shown in \cref{audit}.

%	d) Locate a log message in the audit.log file related to your previous request to the webserver.
%	Explain which information is made available by the log.
%	e) Try to find a corresponding entry in the system’s journal and follow the instruction
%	to display the “complete SELinux message”. Compare the obtained information to the
%	information available in the audit.log file.
%	Once you are finished, stop the webserver: sudo systemctl stop httpd .
\begin{lstlisting}[caption={Relevant logs of the \texttt{audit.log} file}, label={audit}]
type=AVC msg=audit(1588440241.601.380): avc:  denied  { getattr } for  pid=14961 comm="httpd" path="/var/www/html/index.html" dev="dm-0" ino=1180274 scontext=system_u:system_r:httpd_t:s0 tcontext=unconfined_u:object_r:samba_share_t:s0 tclass=file permissive=0
\end{lstlisting}

We can see, that the html object has changed types to \texttt{samba\_share\_t}, as desired. However, since the accessing user is the user \texttt{system\_u} with type \texttt{httpd\_t}, it is not permitted to access the object anymore. Consequently, the permission has been denied by the AVC.

Looking at the system journal in \cref{journal} also shows the attempted access to the object.  

\begin{lstlisting}[caption={Relevant logs in \texttt{journalctl}}, label={journal}]
May 02 19:24:04 DSN-ACSLab-Master setroubleshoot[15479]: SELinux is preventing httpd from getattr access on the file /var/www/html/index.html. For complete SELinux messages. run sealert -l 4898156a-a291-4571-98f4-f89de4d69465
May 02 19:24:04 DSN-ACSLab-Master python3[15479]: SELinux is preventing httpd from getattr access on the file /var/www/html/index.html.
                                                 
      *****  Plugin restorecon (92.2 confidence) suggests   ************************

      If you want to fix the label. 
      /var/www/html/index.html default label should be httpd_sys_content_t.
      Then you can run restorecon.
      Do
      # /sbin/restorecon -v /var/www/html/index.html

      *****  Plugin public_content (7.83 confidence) suggests   ********************
       
      If you want to treat index.html as public content
      Then you need to change the label on index.html to public_content_t or public_content_rw_t.
      Do
      # semanage fcontext -a -t public_content_t '/var/www/html/index.html'
      # restorecon -v '/var/www/html/index.html'

      *****  Plugin catchall (1.41 confidence) suggests   **************************
       
      If you believe that httpd should be allowed getattr access on the index.html file by default.
      Then you should report this as a bug.
      You can generate a local policy module to allow this access.
      Do
      allow this access for now by executing:
      # ausearch -c 'httpd' --raw | audit2allow -M my-httpd
      # semodule -X 300 -i my-httpd.pp
\end{lstlisting}   

Besides the information that access was denied, we also get a suggestion about the right type that should be used. In fact it is the \texttt{http\_sys\_content\_t} type which was assigned before our modification.
Alternatively, the type of the \texttt{index.html} file itself can be changed. 

The logs also indicate an \texttt{sealert} id, which can be used to obtain additional information. Besides what was already made available by the system journal and audit logs, we get the information depicted in \cref{sealert}. 

\begin{lstlisting}[caption={Inspection of alert using \texttt{sealert} (excerpt)}, label={sealert}]
$ sealert -l 4898156a-a291-4571-98f4-f89de4d69465
[...]

Additional Information:
Source Context                system_u:system_r:httpd_t:s0
Target Context                unconfined_u:object_r:samba_share_t:s0
Target Objects                /var/www/html/index.html [ file ]
Source                        httpd
Source Path                   httpd
Port                          <Unknown>
Host                          DSN-ACSLab-Master
Source RPM Packages          
Target RPM Packages          
Policy RPM                    selinux-policy-3.13.1-225.23.fc25.noarch
Selinux Enabled               True
Policy Type                   targeted
Enforcing Mode                Enforcing
Host Name                     DSN-ACSLab-Master
Platform                      Linux DSN-ACSLab-Master 4.13.16-100.fc25.x86_64 #1
                              SMP Mon Nov 27 19:52:46 UTC 2017 x86_64 x86_64
Alert Count                   2
First Seen                    2020-05-02 19:24:01 CEST
Last Seen                     2020-05-02 19:24:01 CEST
Local ID                      4898156a-a291-4571-98f4-f89de4d69465

[...]
\end{lstlisting}

Here we can additionally gather insights into the SELinux status, for instance the policy type and mode, which we previously saw when running the \texttt{sestatus} command in \cref{sec:sestatus}.

\section{Discussion and Conclusion}
In this report, we have outlined the basic concepts of SELinux. The Mandatory Access Control approach that it implements has been interrelated with the traditional Discretionary Access Control concept. Additionally, we have demonstrated the effect of SELinux policies in a brief hands-on part. 



 


\docend
%%% end of document