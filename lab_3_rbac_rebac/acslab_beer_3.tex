\documentclass{semdoc}
% Template: $Id: t01_txt.tex,v 1.7 2000/05/23 12:13:37 bless Exp $
% -----------------------------------------------------------------------------
%epstopdf ermöglicht, dass eps-Dateien durch pdflatex in windows eingebunden werden können
\usepackage{epstopdf}
\usepackage{xcolor}
\usepackage{listings}
\usepackage[backend=biber]{biblatex}
\addbibresource{lab.bib}

\usepackage[acronym]{glossaries}
\newacronym{rbac}{RBAC}{Role-Based Access Control}
\newacronym{rebac}{ReBAC}{Relationship-Based Access Control}
\newacronym{tup}{TUP}{Target User Policy}
\newacronym{trp}{TRP}{Target Resource Policy}
\newacronym{bfs}{BFS}{Breadth First Search}
\usepackage{cleveref}

% Credits to https://latexcolor.com/ <3
\definecolor{amaranth}{rgb}{0.9, 0.17, 0.31}
\definecolor{airforceblue}{rgb}{0.36, 0.54, 0.66}\definecolor{anti-flashwhite}{rgb}{0.95, 0.95, 0.96}

\newcommand*{\Comment}[1]{\hfill\makebox[3.0cm][l]{#1}}%
\lstset{language=python,
    backgroundcolor=\color{anti-flashwhite},
    commentstyle=\color{amaranth},
    keywordstyle=\color{airforceblue},
    numbers=left,
    basicstyle=\ttfamily\small,
	captionpos=b,
	breaklines=true, 
  	escapechar=\&% char to escape out of listings and back to LaTeX
}

% Report Praktikum 
% -----------------------------------------------------------------------------
% Kommentare beginnen mit einem %-Zeichen
\begin{document}
% --> Oberhalb der Linie bitte nichts ändern.
% ---------------------------------------------------------------------------
% \/ \/ \/ \/ \/ \/ \/ \/ \/ \/ \/ \/ \/ \/ \/ \/ \/ \/ \/ \/ \/ \/ \/ \/ \/ 
% Stellen, an denen etwas geaendert werden soll, sind wie hier gekennzeichnet.
% /\ /\ /\ /\ /\ /\ /\ /\ /\ /\ /\ /\ /\ /\ /\ /\ /\ /\ /\ /\ /\ /\ /\ /\ /\ 

%
% ---------------------------------------------------------------------------
% \/ \/ \/ \/ \/ \/ \/ \/ \/ \/ \/ \/ \/ \/ \/ \/ \/ \/ \/ \/ \/ \/ \/ \/ \/ 
% --> Bitte den Titel des Beitrages in die nächste Zeile eintragen:
\title{RBAC \& ReBAC}
%
% --> ... und den Namen des Autors:
\author{Carolin Beer - 1842604}
% /\ /\ /\ /\ /\ /\ /\ /\ /\ /\ /\ /\ /\ /\ /\ /\ /\ /\ /\ /\ /\ /\ /\ /\ /\ 
% -----------------------------------------------------------------------------

% Nicht ändern!
\event{Access Control Systems Lab\\}
\term{Sommersemester 2020}
\supervisor{Prof. Dr. Hannes Hartenstein, Jan Grashöfer, Florian Jacob}


%
\maketitle

\section{Introduction}
In this report, we implement both a \gls{rbac} and a \gls{rebac} policy decision point, i.e. a function which decides whether access to a resource is granted. The underlying data is provided as JSON documents.
Furthermore, we explore the relation of \gls{rbac} and \gls{rebac} by transforming a simple \gls{rbac} instance into a \gls{rebac} instance. 
%---------------------------------------------------------------------------

\section{RBAC}
% \gls{rbac} decides whether or not to grant access to resources based on \emph{Roles}. Users are assigned to certain roles.

In this section, we write a Hierarchical \gls{rbac} instance based on a JSON file \texttt{rbac.json}, which has a data structure with the following five elements:
\begin{itemize}
  \item \texttt{users}: contains all users in the system
  \item \texttt{roles}: contains all roles in the system
  \item \texttt{roleassignment}: for each user, contains a list of roles that they are assigned to
  \item \texttt{rolehierarchy}: for each role, contains a list of roles from which it inherits permissions
  \item \texttt{permissionassignment}: for each resource, contains a dictionary specifying its \texttt{name} as string, and the list of roles \texttt{pa} that are permitted to access it.
\end{itemize}

To evaluate the \gls{rbac} instance, we write a script called \texttt{check\_rbac\_permission.py} which takes two arguments as input. Firstly, the name of the requesting user, and secondly, the name of the requested resource.
We want our implementation to print the access decision to the console.

After parsing the JSON file and the comand line arguments, the function of the script, \texttt{has\_permission()} is called. 
The parsed parameters of the requesting user \texttt{usr}, the requested resource \texttt{res}, and the parsed JSON file \texttt{perm}, are passed as arguments. 

The code \texttt{has\_permission()} is depicted in \cref{permission}.

\begin{lstlisting}[float, language=python, label={permission}, caption={\texttt{has\_permission()} for the RBAC instance}]
# Returns boolean, indicating whether a user has the permission to access the object
def has_permission(usr: str, res: str, perms: dict) -> bool:
    roles = get_roles(usr, perms["roleassignment"])
    targets = get_targets(res, perms["permissionassignment"])

    for role in roles:
        if is_inheriting(role, targets, perms["rolehierarchy"]):
            return True
    return False
\end{lstlisting}

The function first retrieves the roles of the requesting user using \texttt{get\_roles()} and the roles that are granted access to the resource using \texttt{get\_targets()}.
Subsequently, it is checked whether any of the roles of the user inherits any of the roles that are granted access to the resource (targets) in \texttt{is\_inheriting()}.

The semantics of \texttt{is\_inheriting()} are defined in a way that the function returns \texttt{True}, if the role itself is among the target roles as it (reflexively) inherits its own permissions, or if the role (transitively) inherits their permissions of one of them. 
The function keeps track of roles that were explored already, so that each role is visited as most once for each call of \texttt{is\_inheriting()}. 
The complexity of it is therefore in $O(r)$ where $r$ is the number of roles.  

Since a single role that has or inherits the permissions to access a resource suffices to gain permission, \texttt{has\_permission()} returns as soon as a call of \texttt{is\_inheriting()} returns \texttt{True}.
If all of the calls return \texttt{False}, \texttt{has\_permission()} returns \texttt{False} as well.
As \texttt{has\_permission()} calls \texttt{is\_inheriting()} at most once for every role that the user is assigned to, which is bounded by $r$, the overall complexity of a permission check is in $O(r^2)$.
%---------------------------------------------------------------------------

\section{ReBAC}
In this section, we model an instance of \gls{rebac}. 
It differs from \gls{rbac} in the way access decisions are evaluated. 
Instead of having roles, a user graph is used that models relationships.

For a requesting user that wants to access a resource, several policies must be evaluated. 
Firstly, the \gls{trp} of the user that owns a resource, must be considered. It specifies constraints on the length of the path between the requesting and controlling user.
Then, similarly, the \gls{tup} of each target user of a resource must be evaluated.
Since there may be conflicting policy decisions, we also need a conflict resolution mechanism.
Followingly, we will use \texttt{ANY} and \texttt{ALL} as mechanisms, which permit access to a resource if any of the policies is satisfied, or all of the policies are satisfied, respectively.

The data structure of the input JSON file \texttt{rebac.json} differs from the one that we used for the \gls{rbac} instance and now has the following format:

\begin{itemize}
  \item \texttt{users}: contains a list of all users in the system
  \item \texttt{usergraph}: for each user, contains a list of their friends
  \item \texttt{policies}: for each user, contains the \gls{tup} \texttt{tup} and the \gls{trp} \texttt{trp}. Both policies follow the format \texttt{"h"[<|=|>][0-9]}.
  \item \texttt{resources}: for each resource, contains a dictionary with \texttt{name}, controlling user \texttt{controller} and a list of target users \texttt{target}
\end{itemize}

To implement a \gls{rebac} instance, we write a script called \texttt{check\_rebac\_permission.py}. 
The script now takes three arguments: the requesting user, the name of the resource and the conflict resolution policy.

Again, after loading the JSON file and parsing the command line arguments, \texttt{has\_permission()} is called. 
The parameters are the requesting user \texttt{usr}, the requested resource \texttt{res}, the operation to resolve conflicting policies \texttt{op} and the parsed JSON file \texttt{perm}.


\begin{lstlisting}[float, language=python, label={rebac_has_perm}, caption={\texttt{has\_permission()} function for the ReBAC instance}]
# Evaluates whether usr should have access to res. Uses op to resolve conflicting policy decisions.
def has_permission(usr: str, res: str, op: str, perm: dict) -> bool: 
    break_cond = (True if op == "ANY" else False) 
    for pol, tgt in get_policy_per_user(usr, res, perm):
        if eval_policy(pol, usr, tgt, perm) == break_cond:
            return break_cond
    return not break_cond 
\end{lstlisting}

The code in \cref{rebac_has_perm} shows the inner workings of the implementation. 
Using the function \texttt{get\_policy\_per\_user()}, we retrieve a list of tuples, each containing a \gls{trp} or \gls{tup} policy \texttt{pol} as a string, and the user \texttt{tgt} that it must be evaluated for.
We iterate over these tuples and evaluate each of the policies and corresponding user with respect to the requesting user in \texttt{eval\_policy()}. 
A break condition for the loop, which is either \texttt{True} (for \texttt{ANY}) or \texttt{False} (for \texttt{ALL}), is adapted to the chosen conflict resolution operator.
If the policy evaluates to the value defined in that break condition, we can immediately return the overall access decision.

Now let us take a closer look at \texttt{eval\_policy()}, which can be found in \cref{eval-rebac}.

\begin{lstlisting}[float, language=python, label={eval-rebac}, caption={\texttt{eval\_policy()} function for the ReBAC instance}]
# Evaluates a single policy
def eval_policy(pol: str, usr: str, tgt: str, perm: dict) -> bool:
    pat = re.compile('h(>|<|=)([0-9])') 
    m = pat.match(pol)
    if not m or not len(m.groups()) == 2:
      raise Exception(f"invalid policy formatting: {pol}")
    strop, limit = m.groups() 
    op = ops[strop]
    d = shortest_path(usr, tgt, perm["users"], perm["usergraph"])
    # Evaluates whether (d *operator (</>/=)* limit)
    if not op(d, int(limit)):
        return False
    return True
\end{lstlisting}

To evaluate a single policy, we first need to extract the relevant information from the string, this is done in lines 3--8. 
We then have the operator \texttt{op} which is a function that compares two parameters, using one of the operations \texttt{<,=,>}, depending on the policy. Furthermore, \texttt{limit} contains the distance specified by the policy.
To evaluate the policy, we need to compute the distance between \texttt{usr} and \texttt{tgt}. This is done using \gls{bfs} in \texttt{shortest\_path()}. 
\gls{bfs} has a complexity of $O(|E|+|V|)$. Since we have at most $u^2$ edges between $u$ users in the graph, this part of the implementation is in $O(u^2)$.

Having the distance \texttt{d}, we can now evaluate the policy using \texttt{op} and the \texttt{limit} specified by the policy and return the policy decision.
The complexity of the \gls{bfs} dominates the complexity of \texttt{eval\_policy()}.

Since \texttt{has\_permission()} calls \texttt{eval\_policy()} for every \texttt{target} or \texttt{controller} user at most once, and the number of these users is bounded by $u$, the overall complexity is $O(u^3)$. 
However, this complexity could be easily reduced to $O(n^2)$ by reusing results from the \gls{bfs} search.

%---------------------------------------------------------------------------

While we have analyzed a very basic version of \gls{rebac}, it is possible to implement more complex policies.
For instance, it is possible to use different types of connections in the user graph and then to specify policies with respect to the connection type.

We are now exploring the semantics and the syntax of a sample extension of the previously shown code \footnote{While we are not analyzing the code for this extension, you can find the source code that works with and without the extension at \url{https://pastebin.com/Lpy53av5}.}.

We want to have two kinds of relationships, friends and coworkers, for which we both want to create policies. 
Therefore, we need to change the data structure of the JSON file, we now call it \texttt{rebac\_extended.json}. 
It builds on the data structure from the previous \gls{rebac} instance, with the following modifications.

Firstly, the \texttt{usergraph} now contains another layer of nesting. 
For each user, we now have a dictionary, containing the attributes \texttt{friends} and \texttt{coworkers}, each containing the list of users that correspond to that relationship.

Next, the \texttt{trp} and \texttt{tup} policies have a different format, i.e. the policy description language changes. 
Policies may not be empty, and they can contain one or two parts, specifying the distance relation between friends and/or coworkers in the format \texttt{"f"[<|=|>][0-9]} and \texttt{"c"[<|=|>][0-9]}, respectively. 
If two parts are used, they are connected by a \texttt{\&}. 

Using this definition, the semantics of the extended \gls{rebac} instance slightly change. 
The policy evaluation is changes so that it only returns \texttt{True} if all of its parts fulfill the corresponding relations in the user graph. 
Apart from that, policy conflicts can still be resolved using both, \texttt{ANY} and \texttt{ALL}.


\section{RBAC as ReBAC}
Let us now consider the relationship between a non-hierarchical \gls{rbac} and \gls{rebac} instance.
We will usr the example \gls{rbac} instance given in \cref{rbac} and try to transform it into a \gls{rebac} instance, step by step.

\begin{lstlisting}[float, language=bash, label={rbac}, caption={Example of a non-hierarchical RBAC instance}]
{
  "permissionassignment": [
    {
      "pa": [
        "innovation"
      ], 
      "name": "Albania"
    },
    {
      "pa": [
        "human"
      ], 
      "name": "Belgium"
    }
  ], 
  "users": [
    "Michelle", 
    "Lelah"
  ], 
  "roles": [
        "innovation"
        "human"
  ], 
  "roleassignment": {
    "Michelle": [
      "human"
    ], 
    "Lelah": [
      "innovation"
    ]
  }
}
\end{lstlisting}

As we can see, \texttt{Michelle} has the role \texttt{human} and can thus access the resource \texttt{Belgium}. Ananalogeously, \texttt{Lelah} can access \texttt{Albania} through her assigned role \texttt{innovation}.

To ensure that these restrictions are represented in the \gls{rebac} instance, we can add one user in the \gls{rebac} per \texttt{role} in the \gls{rbac}.
Additionally, the existing \texttt{users} from the \gls{rbac} are preserved in the \gls{rebac}. 
To represent the role assignment, we can now introduce relationships between the previous \texttt{role} and the users that got the role assigned, in the \texttt{usergraph}, as shown in \cref{rebac1}.

\begin{lstlisting}[float, language=bash, label={rebac1}, caption={Transferring roles and users from RBAC to a ReBAC instance}]
  "usergraph": {
    "Michelle": [
        "human"    
    ],
    "Lelah": [
        "innovation"
    ],
    "human": [
        "Michelle"
    ],
    "innovation": [
        "Lelah"
    ]
  }, 
  "users": [
    "Michelle",
    "Lelah",
    "human",
    "innovation",
  ], 
\end{lstlisting}

The resources from the \gls{rbac} \texttt{permissionassignment} can be transferred into \texttt{resources} in the \gls{rebac}. Since there are no controlling users in \gls{rbac}, we leave the \texttt{controller} attribute out. 
To restrict access to resources, the previous \texttt{roles} are added as \texttt{target} users. 
Additionally, we need to create one policy per previous \texttt{role}. The \texttt{tup} value is set to \texttt{h=1} for each of them. 
With this policy, it is ensured that accessing users are directly connected to the previous \texttt{role} in the \texttt{usergraph}. 
This can be found in \cref{rebac2}.

\begin{lstlisting}[float, language=bash, label={rebac2}, caption={Transferring access restrictions for resources from RBAC to a ReBAC instance}]
  "resources": [
    {
      "name": "Albania", 
      "target": [ 
        "innovation"
      ]
    }, 
    {
      "name": "Belgium",
      "target": [
        "human"
      ]
    }
  ], 
  "policies": {
    "human": {
      "tup": "h=1"
    },
    "innovation": {
      "tup": "h=1"
    }
  }
\end{lstlisting}


Lastly, the conflict resolution policy should be set to \texttt{ANY} to allow \texttt{users} possessing any of the required \texttt{target} relations, to access the resource.

\section{Discussion and Conclusion}
In the given report, we have seen implementations of both a \gls{rbac} and a \gls{rebac} instance and given an example on how the \gls{rebac} instance could be extended to make it more expressive.
Lastly, we converted a non-hierarchical \gls{rbac} instance into a \gls{rebac} instance for exemplary data. 


\printbibliography

\end{document}
%%% end of document
