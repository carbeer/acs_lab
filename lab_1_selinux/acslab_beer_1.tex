\documentclass{semdoc}
% Template: $Id: t01_txt.tex,v 1.7 2000/05/23 12:13:37 bless Exp $
% -----------------------------------------------------------------------------
%epstopdf ermöglicht, dass eps-Dateien durch pdflatex in windows eingebunden werden können
\usepackage{epstopdf}
\usepackage{listings, xcolor}
\usepackage{cleveref}
\usepackage[backend=biber]{biblatex}

\addbibresource{lab.bib}

% Credits to https://latexcolor.com/ <3
\definecolor{amaranth}{rgb}{0.9, 0.17, 0.31}
\definecolor{airforceblue}{rgb}{0.36, 0.54, 0.66}\definecolor{anti-flashwhite}{rgb}{0.95, 0.95, 0.96}

\newcommand*{\Comment}[1]{\hfill\makebox[3.0cm][l]{#1}}%
\lstset{
    backgroundcolor=\color{anti-flashwhite},
    commentstyle=\color{amaranth},
    keywordstyle=\color{airforceblue},
    basicstyle=\ttfamily\small,
	captionpos=b,
	breaklines=true, 
  	escapechar=\&% char to escape out of listings and back to LaTeX
}

\lstdefinelanguage{Go}{
  % Keywords as defined in the BNF
  morekeywords=[1]{break,default,func,interface,%
    case,defer,go,map,struct,chan,else,goto,package,%
    switch,const,fallthrough,if,range,type,continue,%
    for,import,return,var,select},
  % Special identifiers, builtin functions
  morekeywords=[2]{make,new,nil,len,cap,copy,complex,%
    real,imag,panic,recover,print,println,iota,close,%
    closed,_,true,false,append,delete},
  % Basic types
  morekeywords=[3]{%
    string,int,uint,uintptr,double,float,byte,%
    int8,int16,int32,int64,int128,%
    uint8,uint16,uint32,uint64,uint128,%
    float32,float64,complex64,complex128,%
    rune},
  % Strings : "toto", 'toto', `toto`
  morestring=[b]{"},
  morestring=[b]{'},
  morestring=[b]{`},
  % Comments : /* comment */ and // comment
  comment=[l]{//},
  morecomment=[s]{/*}{*/},
  % Options
  sensitive=true
}

% Report Praktikum 
% -----------------------------------------------------------------------------
% Kommentare beginnen mit einem %-Zeichen
\docbegin
% --> Oberhalb der Linie bitte nichts ändern.
% ---------------------------------------------------------------------------
% \/ \/ \/ \/ \/ \/ \/ \/ \/ \/ \/ \/ \/ \/ \/ \/ \/ \/ \/ \/ \/ \/ \/ \/ \/ 
% Stellen, an denen etwas geaendert werden soll, sind wie hier gekennzeichnet.
% /\ /\ /\ /\ /\ /\ /\ /\ /\ /\ /\ /\ /\ /\ /\ /\ /\ /\ /\ /\ /\ /\ /\ /\ /\ 

%
% ---------------------------------------------------------------------------
% \/ \/ \/ \/ \/ \/ \/ \/ \/ \/ \/ \/ \/ \/ \/ \/ \/ \/ \/ \/ \/ \/ \/ \/ \/ 
% --> Bitte den Titel des Beitrages in die nächste Zeile eintragen:
\title{Reference Monitor (SELinux)}
%
% --> ... und den Namen des Autors:
\author{Carolin Beer - 1842604}
% /\ /\ /\ /\ /\ /\ /\ /\ /\ /\ /\ /\ /\ /\ /\ /\ /\ /\ /\ /\ /\ /\ /\ /\ /\ 
% -----------------------------------------------------------------------------

% Nicht ändern!
\event{Access Control Systems Lab\\}
\term{Sommersemester 2020}
\supervisor{Prof. Dr. Hannes Hartenstein, Jan Grashöfer, Florian Jacob}

%
%
\maketitle

\section{Introduction}
As part of this report, we are developing deeper insights into the understanding of inner workings of the SELinux Reference Monitor as well as custom policy modules. 


\section{The ACS-Tool}
\label{acs-tool}
To learn more about the access restrictions with SELinux, we start off by creating a small executable that tries to read the first line of a given file. 
For this purpose, we use Golang which can be easily compiled to a binary. In \cref{code}, we can see the source code. 
% a)  Write a tool that takes a filename as a parameter and prints out the file’s first line. Note:
% You are free to choose your favorite programming language. For the sake of simplicity, do
% not use an interpreted language but compile a binary.
\begin{lstlisting}[language=Go, caption={Source code of \texttt{freader.go} which reads the first line of a given file and writes it to \texttt{stdout}.}, label={code}]
package main

import "log"
import "bufio"
import "os"

func main() {
    fname := os.Args[1]
    
    f, err := os.Open(fname)
    if err != nil {
        log.Fatal(err)
    }

    defer func() {
        if err = f.Close(); err != nil {
            log.Fatal(err)
        }
    }()

    s := bufio.NewScanner(f)
    s.Scan()
    log.Println(s.Text())
}
\end{lstlisting}

Next, we compile the code which leaves us with a executable named \texttt{freader}, and create a dummy text file named \texttt{acs.txt} for testing purposes as shown in \cref{init}.


\begin{lstlisting}[caption={Building the \texttt{freader} script and creating the \texttt{acs.txt} file.}, label={init}]
$ go build freader.go 
$ echo "Hello ACS-Lab!" > acs.txt
\end{lstlisting}

% b) Run the ACS-Tool on the acs.txt -file as well as on /etc/shadow and explain the result.
Now, we are all set up to run a first test by calling our \texttt{freader} executable with the \texttt{acs.txt} file. 
As expected, our executable reads the first line and writes it to the command line.
Next, we try to read the contents of \texttt{/etc/shadow} which is used to store password hashes. 
Luckily, the \texttt{freader} binary is unable to read the contents of this file. 
Taking a look at the output of the \texttt{ls -lZ} commands in \cref{1b}, the reason for this behavior becomes clear.
The Discretionary Access Control (DAC) allows read access to the \texttt{acs.txt}, but not for \texttt{/etc/shadow}. 
The Mandatory Access Control (MAC) implemented by SELinux on top of the DAC does not restrict access of the \texttt{freader} to the \texttt{acs.txt} file any further, therefore it has been able to successfully read the content.

\begin{lstlisting}[caption={Accessing different files using the \texttt{freader} binary.}, label={1b}]
$ ./freader acs.txt 
2020/05/11 23:18:25 Hello ACS-Lab!
$ ./freader /etc/shadow
2020/05/11 23:20:12 open /etc/shadow: permission denied
$ ls -lZ acs.txt 
-rw-r--r--. 1 student vboxsf unconfined_u:object_r:user_home_t:s0 15 May 14 20:45 acs.txt
$ ls -lZ /etc/shadow
----------. 1 root root system_u:object_r:shadow_t:s0 1323 May  2 19:52 /etc/shadow
$ ls -lZ freader 
-rwxrwx---. 1 student vboxsf unconfined_u:object_r:user_home_t:s0 1946890 May 11 23:17 freader
\end{lstlisting}

\section{Writing a Policy}
In the following, we learn about policy modules, analyze their basic components, and put them to use.

\subsection{Understanding the Components of a Policy Module}
% a) Explain the purpose of the different files ( acs.te , acs.if and acs.fc ).
The SELinux project maintains a \emph{reference policy} which can be used as-is for various systems or may serve as a basis for system administrators to build their own custom policy \cite{ref-policy}.

The reference policy is designed in a modular fashion, meaning that custom modules for applications can be added where needed.
Modules generally contain three different kinds of files \cite{Hertzog2015}:

\begin{itemize}
  \item \texttt{.te}: type enforcement file, defines the rules of the module. It contains the core logic and is mandatory in any module. 
  \item \texttt{.fc}: file context file, specifies the types that are assigned to files that are addressed by the module. Optional.
  \item \texttt{.if}: interface file, contains the \emph{interfaces} of the module. These are used for interaction between different modules. Optional. 
\end{itemize}

% b) Explain the acs.te file line by line. Hint: If you do not know the effects of a specific line,
% comment out that single line, continue with the task and analyze the consequences of that
% line missing.

In \cref{acs-te} an example of a type enforcement file is shown.
To improve our understanding of this component of the policy module, we will work through it line by line.

\begin{lstlisting}[caption={Contents of the \texttt{acs.te} type enforcement file.}, label={acs-te}, numbers=left]
policy_module(acs, 1.0.0)

gen_require(`
	role unconfined_r;
	type unconfined_t;
')

type acs_t;
type acs_exec_t;

domain_type(acs_t)
domain_entry_file(acs_t, acs_exec_t)

role unconfined_r types acs_t;
type_transition unconfined_t acs_exec_t : process acs_t;

userdom_use_user_terminals(acs_t)

type acs_file_t;
files_type(acs_file_t)
\end{lstlisting}

In the first line of the file, the name and version number of the module are stated.
This is a requirement for all modules.

The \texttt{gen\_require} macro keyword is used to specify the set of external policy components, the role \texttt{unconfined\_r} and the type \texttt{unconfined\_t}, which are required, but not defined by the module itself.

This is followed by declarations of two new types \texttt{acs\_t} and \texttt{acs\_exec\_t}.
By calling \texttt{domain\_type()} on the newly created type \texttt{acs\_t}, we mark it as domain type. 

Subsequently, the \texttt{acs\_t} domain is assigned an entry point using \texttt{domain\_entry\_file()}.
In this case, this is the \texttt{acs\_exec\_t} type.

Line 14 specifies that the role \texttt{unconfined\_r} is allowed to assume our new domain \texttt{acs\_t}.

With the given \texttt{type\_transition} in line 15, a process transition, as indicated by the \texttt{process} object class following the colon, is defined.
It denotes the transition that a process with some source type, here \texttt{unconfined\_t} should undergo when executing an object of some target type, here \texttt{acs\_exec\_t}. 
The type of the process after the transition, if allowed by the policy, is called the default type. 
In the given example, this is the \texttt{acs\_t} type.
    
\texttt{userdom\_use\_user\_terminals()} is an interface defined by the reference policy with a domain as argument. It allows the given domain, here \texttt{act\_t}, to interact with the user through a terminal.
It is therefore mainly needed for interactive applications. 

Lastly, a new type \texttt{acs\_file\_t} is declared and made applicable to files using \texttt{files\_type()}. 

\subsection{Loading and Using a Custom Policy Module}
% Compile the policy module (use make to run the linked Makefile) and load it into SELinux
% ( semanage module -a acs.pp ). Using semanage you can verify that the module is loaded successfully.
% Now, label the executable with the new type label ( acs_exec_t ) using chcon .
Now after compiling the module with \texttt{make} and loading the policy into SELinux using \texttt{semanage}, as shown in \cref{make}, we can make use of our new labels.

\begin{lstlisting}[caption={Compiling the \texttt{acs} policy and loading it into SELinux.}, label={make}]
$ make 
Compiling targeted acs module
/usr/bin/checkmodule:  loading policy configuration from tmp/acs.tmp
/usr/bin/checkmodule:  policy configuration loaded
/usr/bin/checkmodule:  writing binary representation (version 17) to tmp/acs.mod
Creating targeted acs.pp policy package
rm tmp/acs.mod.fc tmp/acs.mod
$ sudo semanage module -a acs.pp
$ sudo semanage module -l | grep "acs"
acs                       400       pp    
\end{lstlisting}

We assign a new label to the \texttt{freader} binary using \texttt{chcon}. 
In \cref{chcon}, we can see that the type change was successful, as the output of \texttt{ls -lZ} shows the new type \texttt{acs\_exec\_t} in the security context.
Trying to run the \texttt{freader} on the \texttt{acs.txt} file now yields a permission error, too.

\begin{lstlisting}[caption={Changing the label of the \texttt{freader} binary.}, label={chcon}]
$ ls -lZ freader 
-rwxrwx---. 1 student vboxsf unconfined_u:object_r:user_home_t:s0 1946890 May 11 23:17 freader
$ sudo chcon -t acs_exec_t ./freader 
$ ls -lZ freader 
-rwxrwx---. 1 student vboxsf unconfined_u:object_r:acs_exec_t:s0 1946890 May 11 23:17 freader
$ ./freader acs.txt 
2020/05/14 20:57:48 open acs.txt: permission denied
\end{lstlisting}

% c) Execute your ACS-Tool and explain why you cannot access the test file anymore.
Based on the type transition rule that has been specified in our policy module, the process transitions to the \texttt{acs\_t} type when executing the \texttt{freader} binary of type \texttt{acs\_exec\_t}. 

Since the newly defined \texttt{acs\_t} type only possesses the very minimal default privileges, it is not granted read access to our \texttt{acs.txt} file anymore.

\subsection{Modifying the Type Enforcement File}
Now, we need to modify the type enforcement file to elevate the privileges of the \texttt{acs\_t} type to make our executable functional again.
Following the philosophy of SELinux we also want to avoid giving applications more privileges than needed. 
Therefore, to prevent the executable from accessing other files on the system, we use the dedicated \texttt{acs\_file\_t} type to label files that shall be accessible by the \texttt{freader}.

% d) Edit the ACS policy module to allow programs in the acs_t domain to list directories (search
% files) and read files labeled with the acs_file_t type. Label the acs.txt file accordingly
% and use the ACS-Tool to access it.


\begin{lstlisting}[caption={Additions to the \texttt{acs.te} file.}, label={te-new}, numbers=left]
gen_require(`
        ; [...]	
        type user_home_t;
')
; [...]
allow acs_t acs_file_t:file {open read};
allow acs_t user_home_t:dir search;
\end{lstlisting}

In \cref{te-new} the required additions to the \texttt{acs.te} are illustrated. 
Line 6 allows processes of type \texttt{acs\_t} to read and open files of type \texttt{acs\_file\_t}. 
For the executable to successfully find the file, it also requires search access to the directory. 
This is specified in line 7.
Since the \texttt{user\_home\_t} directory type, which serves as default type of directories on the home path, was not previously part of the type enforcement file, we need to add it to the \texttt{gen\_require} block.

Now it remains to relabel the \texttt{acs.txt} file with the \texttt{acs\_file\_t} type using \texttt{chcon}, as shown in \cref{relabel-txt}.
Trying to run the \texttt{freader} on it now yields the file content, as desired.

\begin{lstlisting}[caption={Changing the label of the \texttt{acs.txt} file.}, label={relabel-txt}]
$ sudo chcon -t acs_file_t acs.txt 
$ ./freader acs.txt
2020/05/14 23:57:26 Hello ACS-Lab!
\end{lstlisting}


% After changing the label of the acs.txt file, one could use the ACS-Tool to access the file. Relabel
% using the acs_file_t type. Warning: Make sure to restore the default label using
% restorecon after your test.

\subsection{Mandatory and Discretionary Access Control}
% e) Explain why it is still not possible to use the ACS-Tool to access /etc/shadow .

Recalling the experiment from \cref{acs-tool}, we could now try to relabel the \texttt{/etc/shadow} file as well and see whether we can read from it.

\begin{lstlisting}[caption={Relabeling \texttt{/etc/shadow} and trying to read it with the binary.}, label={shadow}]
$ sudo chcon -t acs_file_t /etc/shadow
$ ls -lZ /etc/shadow
----------. 1 root root system_u:object_r:acs_file_t:s0 1323 May  2 19:52 /etc/shadow
$ ./freader /etc/shadow
2020/05/15 11:10:58 open /etc/shadow: permission denied
$ restorecon /etc/shadow
\end{lstlisting}

As we can see from \cref{shadow}, it is still not possible to access \texttt{/etc/shadow} after the relabel operation.
The reason for this is simple. 
SELinux does not substitute the DAC mechanisms embedded in Linux, but works on top of them.
This means that the SELinux MAC policies are only ever evaluated, when the DAC would grant access. 
As we can see from the output of \texttt{ls -lZ}, the DAC prevents read access already. 
Thus, changes in the policy do not have any effect on the access rights of the \texttt{freader} to the \texttt{/etc/shadow} in this case.

\subsection{Putting the Given Example Into Perspective}
% f) Provide a brief discussion on whether or not the ACS-Tool example is realistic.
While our hands-on part has clearly shown the impact of policies, it only partially represents how application access rights are configured in reality.
The reference policy already provides a large set of labels and interfaces which can be used off the shelf and are more handy than configuring label access rights by hand. 
For instance, if we were to create a new file type for logging purposes, we could simply use the interface \texttt{logging\_log\_file()} to automatically assign rights to relevant logging types. 
However, for customized applications it may still be useful, to explicitly assign rights, as we did in our example.

\section{SELinux as Reference Monitor}
% 1. Evaluable → small enough to be subject to analysis
% 2. Always invoked → no alterative access method
% 3. Tamper-proof → mechanism cannot be altered
% Furthermore, the extended Reference Monitor concepts have been introduced (c.f. Lecture 1),
% i.e. Authorization Database, Monitor Interface and Audit Trails.

% a) Map the extended Reference Monitor concepts to their Linux counterpart.

We now take a look at the extended Reference Monitor concepts – namely the authorization database, monitor interface and audit trails – and map them to their Linux counterpart.

The monitor interface in Linux consists of \emph{system calls} that are used to look up privileges whenever a process requests access to some resource.
By default, the authorization in Linux is based solely on DAC. The resulting authorization database is represented by \emph{inodes} storing \emph{access control matrices}. They contain the group and owner of a file, and the corresponding permissions, i.\,e. read, write or modify for regular files, depending on the user type. 
Lastly, the file \texttt{/var/log/audit/audit.log} contains the audit trails.
With Linux Security Modules (LSM), access control can be extended. The corresponding implementations, such as SELinux, are integrated through \emph{hooks}.
\cite{Jaeger2008}

% b) Comment on whether you think SELinux complies with the Reference Monitor requirements.

Considering the three reference monitor implementation requirements of being \emph{evaluable}, \emph{always invoked} and \emph{tamper-proof}, SELinux seems to be compliant with all of them. 
While it is in general difficult to formally verify the correctness of security enforcement mechanisms, SELinux should be easy to test and therefore evaluate, especially due to its modular design. 
Furthermore, due to the placement of hooks, SELinux is always invoked whenever privileges are requested and granted by DAC.
Finally, SELinux is tamper-proof because it uses policies that cannot be modified by system users, as opposed to DAC. 

\section{Discussion and Conclusion}
Throughout this report we have gained insights into the inner workings of SELinux policy modules. In a hands-on part, we have learnt how labels and affect access privileges and how SELinux is entangled with DAC. 
The practical part was rounded off with a brief theoretical part about the concept of a reference monitor.

\printbibliography

\docend
%%% end of document
